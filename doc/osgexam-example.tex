\documentclass[
  xlsfile=./Zulassung/students.xlsx,
  prefix=pn,         % for "Programming and Nonsense"
  outpath=mypath,  % target directory for exams (default: "pdf")
  %group=true,
  %seat=true,
  %development=answers,  % generate standard solution in developer mode 
  development=true,        % uncomment during (early phase of) development
  lang={de,en},      % possible values: de,en,{de,en},{en.de} - first language determines main language
  %solutiontime=45, % same as preamble command \solutionTime
  %quotes=ascii,    % shorthands for quotes, possible values: babelshorthands => "`phrase to quote"'
                     %                                         ascii => "phrase to quote"
                     %                                         latex => ``LaTeX standard behavior''
  % impolite   % avoid few polite phrases to shorten title page
   ]{osgexam}

  \course[AuN]{\cdeen{Programmierung und Unsinn}{Programming and Nonsense}}
  \term[WS22/23]{WS 2022/2023}
  %\solutionTime{120}
  \addextrapoints[Extra]{3}

\begin{document}
 
\begin{questions}
    
    \question[23] \deen{Was ist die Antwort auf die Frage nach dem Universum,
      dem Leben und den ganzen Rest?}{What ist the answer to the question for
      life, universe, and everything?}

  \fillin[42]

  \question
  \deen{Alle Multiple-Choice-Fragen haben genau eine korrekte Antwort.
    Wenn  Sie mehr als eine markieren, erhalten Sie keine Punkte. Bei Korrekturbedarf machen Sie alle Kästchen für die
    jeweilige Frage ungültig und schreiben das Wort "`KORREKT"' neben die angedachte Antwort.}{All multiple choice
    questions have exactly one correct answer.
    If you mark more than one, you will get no points. If you need to correct
    yourself, invalidate all choices of this particular question and write the word "`CORRECT"' beside your
    intended choice.}
  \begin{parts}
    \part[1]
    \deen{Sind Sie sicher?}{Are you sure?}
    \begin{checkboxes}
      \choice \sdeen{ja}{yes}
      \correctchoice \sdeen{nein}{no}
    \end{checkboxes}
    \bonuspart[1]
    \deen{Wirklich?}{Really?}
    \begin{checkboxes}
      \choice \sdeen{ja}{yes}
      \correctchoice \sdeen{nein}{no}
    \end{checkboxes}
  \end{parts}

  \question[2]
  % \xdeen{<deutsch>}{<englisch>}{<1. Ersetzungsparameter>}... erlaubt gemeinsame Elemente (z.B. Formeln) in beiden
  % Sprachteilen und sorgt so für Konsistenz zwischen den Sprachversionen:
  \xdeen{Gegeben eine Taskmenge @1 mit @2. Wieviele Tasks sind in @1?}{Given a task set @1 with @2. How many task are in
    @1?}{$\mathbf T$}{${\mathbf T}=\{\tau_{1},\tau_{2},\tau_{2}\}$}

  \begin{solution}
    $|{\mathbf T}| = 3$
  \end{solution}

  \titledquestion{\sdeen{Netter Titel}{Nice title}}[1]
  \deen{Haben Sie einen besseren Titel?}{Do you know a better title?}
  
  \bonusquestion[1]
  \deen{Welche Aussage hören Prüfer am häufigsten in mündlichen Prüfungen?}{What phrase are the examiner hearing most
    often in oral exams?}
  \begin{solution}
    \textenglish{"`I don't know."'}
  \end{solution}

\end{questions}

\end{document}