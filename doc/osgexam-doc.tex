\documentclass[
load-preamble
]{cnltx-doc}

\selectlanguage{ngerman}
%\usepackage{readprov}
%\usepackage{xspace}
\setcnltx{
  class=osgexam,
  subtitle={Eine Klasse für Examen an der Professur Betriebssysteme der TU~Chemnitz},
  % build-title,
  % date=2021-06-20,
  % version=0.2, 
  authors=Matthias Werner,
  email=matthias.werner@informatik.tu-chemnitz.de,
  % abstract={Die \LaTeX-Klasse \cls*{exam} von \textsf{Philip Hirschhorn} eigenet sich exellent zur Erstellung von
  %   Klausurdokumenten mit \LaTeX. Die Klasse foo %\osgexam\
  %   baut auf der \cls*{exam}-Klasse auf, passt sie an die
  %  Anforderungen für Klausuren an, die an der Professur für Betriebssysteme der TU~Chemnitz existieren und stellt
  %  einige Erweiterungen sowie ein Buildsystem für verschiedene Versionen der Klausur zur Verfügung zur Verfügung}
}


\begin{document}
%\GetFileInfo{osgexam.cls}
%\title{Die osgexam-Klasse}
%\author{Matthias Werner}
\maketitle


% \end{abstract}
\part{Einführung}
\section{Inhalt}
Die \osgexam-Klasse hat zwei Funktionen
\begin{enumerate}
  \item Sie erweitert die \cls{exam}-Klasse von \textsf{Philip Hirschhorn} entsprechend den Anforderungen der
    Professur Betriebssysteme der TU~Chemnitz
  \item Sie dient als Buildsystem für verschiedene Versionen einer Klausur, einschließlich der Generierung eines
    kompletten Klassensatzes von Klausuren aus den Daten einer Excel-Datei.
\end{enumerate}
Die \osgexam\ funktioniert nur mit \texttt{lualatex}, wobei andere Programm aufgerufen werden. Daher muss die Übersetzung
stets mit der Aktivierung der Shell-Escape-Funktion vorgenuommen werden, also z.B.:\medskip

\noindent\verb!>lualatex -shell-escape myexam.tex!

\medskip
Alternativ kann \texttt{latexmk} mit der im Paker zur Verfügung gestellten \texttt{latexmkrc}-Datei genutzt werden:
\medskip

\noindent\verb!>latexmkrc myexam.tex!
\medskip


\end{document}
