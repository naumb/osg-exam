\documentclass[
load=osgexam,
babel=ngerman
]{skdoc}
\microtypesetup{disable=true}

%\changes{0.1}{Initiale Version}
%\changes{0.2}{Umstellung auf pdfkeys, Anpassung an neue Version von openpyxl, }
%\changes{0.2a}{Dokumentation}

\usepackage{enumitem}
\usepackage{readprov}
\usepackage{listings}
\lstloadlanguages{[LaTeX]TeX}
\lstset{language=TeX,basicstyle=\small\ttfamily,
  morekeywords={documentclass,xdeen},
frame=leftline}


\ReadFileInfos{osgexam.cls}
\package{osgexam}
\version{\fileversion}
\makeatletter
\let\mygobble=\@gobble
\deftripstyle{skdoc}%
    {}{}{}%
    {\small Die~\textbf{\pkg*{\@package}}-Klasse,~v\@version}{}{\small\pagemark}
%\cls{osgexam}
\makeatother
\author{Matthias Werner}
\email{matthias.werner@informatik.tu-chemnitz.de}
\title{Die \thepkg-Klasse}
\subtitle{Eine Klasse für Examen an der Professur Betriebssysteme der TU~Chemnitz}

\begin{document}
\maketitle
\begin{abstract}
    Die \LaTeX-Klasse \pkg{exam} von \textsf{Philip Hirschhorn} eigenet sich exellent zur Erstellung von
    Klausurdokumenten mit \LaTeX. Die Klasse \thepkg\  baut auf der \pkg{exam}-Klasse auf, passt sie an die
    Anforderungen für Klausuren an, die an der Professur für Betriebssysteme der
    TU~Chemnitz existieren, und stellt einige Erweiterungen sowie ein
    Buildsystem für verschiedene Versionen der Klausur zur Verfügung. 
\end{abstract}

\tableofcontents\clearpage


\section{Einführung}
\subsection{Zweck}
Die \cls{osgexam}-Klasse hat zwei Funktionen
\begin{enumerate}
  \item Sie erweitert bzw.\ konfiguriert die \pkg{exam}-Klasse von \textsf{Philip Hirschhorn} entsprechend den
    Anforderungen der Professur Betriebssysteme der TU~Chemnitz
  \item Sie dient als Buildsystem für verschiedene Versionen einer Klausur, einschließlich der Generierung eines
    kompletten Klassensatzes von Klausuren aus den Daten einer Excel-Datei.
\end{enumerate}
\subsection{Systemvoraussetzungen}
Die \thepkg-Klasse verlangt Lua\LaTeX\ und läuft nicht mit pdfTeX. Sie baut auf der Klasse \pkg{exam}\ auf, daher muss
in der \LaTeX-Installation diese Klasse sowie alle Pakete, von denen \pkg{exam}
abhängt, vorhanden sein.
 
Für den Seriendruck ist \file{latexmk} sowie ein Python3 mit installierten
\file{openpyxl} notwendig. Letzteres kann z.\;B.\ über

\begin{verbatim}
>pip3 install openpyxl
\end{verbatim}
in der Shell installiert werden.

\subsection{Dokumenttypen und Modi}
\label{sec:doctypes}
Mit dieser Klasse können verschiedene Arten von Dokumenten erzeugt werden, die mit verschiedenen Modi verbunden sind:
\begin{itemize}[nosep]
  \item Nichtpersonalisiertes (also leere) Aufgabenblatt (Modus \opt{empty})
  \item Personalisiertes Aufgabenblatt mit vorgegebenen Daten (Modus \opt{single})
  \item Personalisiertes Aufgabenblatt mit Dummy-Daten (Modus \opt{dummy})
  \item Serie von personaliserten Aufgabenblättern entsprechend den Daten in einer Exceldatei (Modus \opt{series})
  \item Musterlösung  (Modus \opt{answers})
\end{itemize}
Es ist vorgesehen, den jeweiligen Modus interaktiv zu setzen.
Während der Entwicklung einer Klausur ist die Interaktivität vermutlich störend, daher kann sie mit der
\opt{developmet}-Option (siehe Abschnitt~\ref{sec:options}) abgeschaltet werden. Jedoch stehen dann einige der Modi nicht mehr zur
Verfügung.

Um auch während der Entwicklung auf diese Modi zurückgreifen zu können, kann im Quelldokument noch vor
\cs{documentclass} der jeweilige Modus mit einem \TeX-Befehl definiert werden.
Beispielsweise kann der \opt{series}-Modus auf folgende Weise unter Umgehung des Auswahlmenüs erzwungen werden:
\begin{lstlisting}
    \def\mode{series}
    \documentclass{osgexam}
    % ...
\end{lstlisting}

\section{Klassenoptionen}
\label{sec:options}
Die Klasse übernimmt alle Options von \pkg{exam}, fügt aber eine Reihe neuer
hinzu. Ausserdem werden einige Option von \pkg{exam} in den einzelnen Modi überschrieben.

\subsection*{Modi und Dateien}
\Option{development}\WithValues{true,false,answers}\AndDefault{true}
Der Buildmechanismus wird bei den Werten \opt{true} oder \opt{answers} deaktiviert; das Dokument kann also wie ein
normales \LaTeX-Dokument ohne Nutzerinteraktion übersetzt werden. Der Wert \opt{answers} entspricht dem Setzen der
\pkg{exam}-Option \opt{answers}, es wird also die Musterlösung generiert.

Wird die Option nicht oder auf den Wert \opt{false} gesetzt, wird interaktiv die Art des zu erzeugenden Dokumentes
abgefragt (siehe Abschnitt~\ref{sec:doctypes}).
\medskip

\Option{prefix}\WithValues{\meta{Präfix}}
Setzt den Präfix für die Ausgabedatei(en). Da der Dateiname einer Ausgabedatei durch den Dokument\emph{typ} (siehe Abschnitt~\ref{sec:doctypes}) festgelegt
wird, hilft das Setzen eines Präfix, den Zusammenhang mit dem Quelldokument herzustellen.
\medskip

\Option{xlsfile}\WithValues{\meta{\file{Excel-Datei}}} Excel-Datei für die Seriengenerierung von einem Klassensatz von
Aufgabenbättern, siehe Abschnitt~\ref{sec:series}.
\medskip

\Option{outpath}\WithValues{\meta{Pfad}}\AndDefault{pdf}
Dateipfad zu einem Verzeichnis für die generierten Dateien (außer im Developer-Mode). Damit kann das Verzeichnis mit den
Aufgabenstellungen von den generierten Dateien freigehalten werden. Dies kann insbesondere bei der Seriengenerierung die
Übersichtlichkeit steigern.

Der Pfad wird relativ zum aktuellen Arbeitsverzeichnis interpretiert. Wenn ein absoluter Pfad angegeben werden soll,
muss er wie üblich mit einem Schrägstrich (/) beginnen.
\medskip

\subsection*{Sprache}
\Option{lang}\WithValues{de,en,\{de{,}en\},\{en{,}de\}}\AndDefault{de}
Legt die Sprache(n) des Dokuments fest. Es sind nur Deutsch (\opt{de}) und Englisch (\opt{en}) vorgesehen. In
Kombination mit den Sprachmacros (siehe Abschnitt~\ref{sec:macro:lang}) können aus der gleichen Quelldatei verschiedene
Sprachvarianten generiert werden. Werden beide Sprachen angegeben, müssen
geschweifte Klammern gesetzt werden. Die erste Sprache in der Klammer ist dann
die \emph{primäre} Sprache.\medskip

\Option{quotes}\WithValues{babelshorthands,ascii,latex}\AndDefault{babelshorthands}
Mit dieser Option wird gesteuert, wie Anführungszeichen eingegeben werden sollten.
\begin{itemize}[nosep]
  \item [\opt{babelshorthands}] Anführungszeichen werden als Shorthand entsprechend der (deutschen) Einstellung für
    \pkg{babel} erwartet, also als \verb!"`! / \verb!"'!. Babel-Shorthands funktionieren unabhängig von der
    Spracheinstellung, also auch in englischen Textabschnitten.
  \item [\opt{ascii}] Anführungszeichen werden als einfache ASCII-Anführungszeichen erwartet (\verb!"!/\verb!"!). 
  \item [\opt{latex}]  Der Standardmechanismus von \LaTeX (\verb!``!/\verb!''!) wird angenommen. Diese Einstellung
    erfolgt auch, wenn ein sonstiger Wert gewählt oder die Option ganz weggelassen wird.
\end{itemize}
\Notice{Bei Wahl von \opt{babelshorthands} oder \opt{ascii} wird (ähnlich wie in \pkg{babel}) der Kategoriecode der
  Anführungszeichen geändert, was ggf. mit anderen Paketen interferieren kann.}
\medskip

\subsection*{Gestaltung der Titelseite}
\Option{solutiontime}\WithValues{\meta{Zeit in Minuten}}\AndDefault{90}
Legt die zur Verfügung stehende Lösungszeit fest. Der hier angegebene Wert wird im Standardtitelblatt verwendet.
Statt als Option kann die Lösungszeit auch über das Macro \Macro{solutionTime} in der Preamble gesetzt werden.
\medskip

\Option{passedwith}\WithValues{\meta{Punktegrenze für das Bestehen in \%}}\AndDefault{50}
Es wird festgelegt, ab wieviel Prozent der Punkt die Klausur bestanden ist. Der berechnete Wert wird auf der Titelseite
angegeben.
\medskip

\Option{seat}\WithValues{no,title,xml} Steuert die Angabe eines Sitzplatzes.Wird eine Klausur parallel in mehreren Räumen
geschrieben, empfiehlt es sich, hier (auch) den Raum anzugeben.
\begin{itemize}[nosep]
  \item [\opt{no}] Auf der Titelseite wird kein Sitz angegeben.
  \item [\opt{title}] Auf der Titelseite wird ein Feld für die Angabe eines
    Sitzplatzes vorgesehen. Dieses ist zunächst leer, so dass der Sitzplatz
    später handschriftlich hinzugefügt werden kann. Beim \opt{single}-Modus
    wird ein Sitzplatz abgefragt, aber auch dieser kann leer gelassen werden.
  \item [\opt{latex}]  Der Standardmechanismus von \LaTeX (\verb!``!/\verb!''!) wird angenommen. Diese Einstellung
    erfolgt auch, wenn ein sonstiger Wert gewählt oder die Option ganz weggelassen wird.
\end{itemize}

\medskip
\Option{group}Gibt die Gruppe auf der Titelseite an.
\Notice{Die Nutzung des Mechanismus zum Ein-/Ausschließen von Gruppen
  (siehe Abschnitt~\ref{sec:groups}) und
  die Anzeige einer Gruppe auf der Titelseite sind unabgängig voneinander.}
\medskip

\Option{impolite}Die Willkommens-Formel und der Erfolgswunsch wird unterdrückt. Damit kann ggf. etwas Platz auf der
Titelseite der Klausur gespart werden, falls zusätzliche oder geänderte Klausurregeln (siehe Abschnitt~\ref{sec:macro:title})
zusätzlichen Platz beanspruchen. 
\medskip


\section{Die verschiedenen Dokumenttypen}
Wenn ein Dokument mit der \thepkg-Klasse übersetzt wird, ohne dass die \opt{development}-Option gesetzt ist, erscheint
im Terminal ein Textmenü:
\begin{verbatim}
******************************************************
 No mode provided.
 Please enter a proper mode (default is 'e') 
 - [e]mpty exam
 - [a]nswers (solution)
 - [p]ersonalized exam series from MS Excel file
 - [i]ndividually personalized exam
 - [d]ummy personalized exam

 - [q]uit 

\mode =
\end{verbatim}

Durch Auswahl des entsprechenden Eintrags wird die Übersetzung erneut~-- diesmal im ausgewählten Modus~-- gestartet.

\subsection{Aufgabenblatt: Modi \opt{empty}, \opt{single} und \opt{dummy}}
Bei den Modi  \opt{empty}, \opt{single} und \opt{dummy} wird jeweils eine einzelne Klausur generiert.
Die Modi unterscheiden sich durch die Personalisierung der Klausur. Während in \opt{empty} keine Personaliserung
vorgenommen wird, wird in \opt{dummy} die Klausur mit generischen Daten (Max Mustermann, \ldots) und im \opt{single}
(Menüpunkt ``individually personalized exam") mit konkreten Personendaten personalisiert. Letztere werden interaktiv
abgefragt.

Der Name der Ausgabedatei ist \meta{prefix}-\meta{Studenten-ID}\file{.pdf} in Fall von \opt{single}, und
\meta{prefix}-\file{dummy.pdf} bzw. \meta{prefix}-\file{empty.pdf} im Fall von \opt{single} bzw. \opt{empty}. Die
Ausgabedatei wird ins Verzeichnis \meta{outpath} geschrieben.

\subsection{Musterlösung: Modus \opt{answers}}
In deisem Modus wird eine Musterlösung erstellt, ähnlich wie mit der \opt{answers} in der Klasse \pkg{exam}.
Die Musterlösung enthät auf allen Seiten ein Wasserzeichen mit dem Wort ``Musterlösung'' bzw. ``Standard Solution'', je
nachdem welches die (primäre) Sprache des Dokuments ist.

\subsection{Seriendruck: Modus \opt{series}}
\label{sec:series} 
Um den Seriendruck von personaliserten Klausuren zu nutzen (Modus \opt{series}), muss eine Excel-Datei mit in der
\opt{xlsfile}-Option angegebenen Namen und den entsprechenden Daten vorbereitet werden.
\DescribeFile{xlsfile} Die Datei muss auf dem ersten Arbeitsblatt eine Tabelle mit den Identifikationsnummer der
Studierenden, ihren Namen und Vornamen, sowie ggf.\ den zugewiesenen Platznummern enthalten. Die Spalten der Tabelle
werden über einen Tabellenkopf identifiziert, wobei die Kopfeinträge unterschiedlich lauten können:
\begin{itemize}[nosep]
  \item Identifikationsnummer: ``\emph{Matrikelnummer}'', ``\emph{Matr-Nr}'', ``\emph{Matrikel}'', ``\emph{id}'', ``\emph{student id}''
  \item Name: ``\emph{Name}'', ``\emph{surname}'', ``\emph{family name}''
  \item Vorname: ``\emph{Vorname}'', ``\emph{first name}'', ``\emph{forename}'', ``\emph{given name}''
  \item Platz: ``\emph{Platz}'', ``\emph{Sitz}'', ``\emph{seat}'', ``\emph{place}''
  \item Gruppe: ``\emph{Gruppe}'', ``\emph{group}''
\end{itemize}
Es werden alle Einträge unter der Kopfzeile bis zur ersten Leerzeile verarbeitet, d.\;h. entsprechend viele Dateien in
\meta{outpath} angelegt. Diese sind wieder nach dem Schema \meta{prefix}-\meta{Studenten-ID}\file{.pdf} benannt.
Man beachte, dass das Erzeugen eines Klassensatzen von Klausuren durchaus einige Zeit in Anspruch nehmen kann.

\section{Macros zur Klausurgestaltung}
\subsection{Sprache}
\label{sec:macro:lang}
Die \thepkg-Klasse stellt den Erstellern von Klausuren eine Reihe von Macros zur Sprachunterstützung bereit.

\DescribeMacro\deen{\meta{deutsche Textvariante}}{\meta{englische Textvariante}} Wenn nur eine Sprache bei der
\opt{lang}-Option angegeben ist, wird diese ausgegeben. Sind zwei Sprachen ausgewählt, wird zunächst die Erstsprache
ausgegeben und anschließend die Textvariante der Zweitsprache in der Folgezeile in einer alternativen Farbe.
\medskip

\DescribeMacro\sdeen{\meta{deutsche Textvariante}}{\meta{englische Textvariante}}Dieses Macro funktioniert wie
\Macro{deen}, nur dass statt eines Zeilenumbruchs ein Schrägstrich (``/'') die Textvarianten trennt.
\medskip

\DescribeMacro\ldeen{\meta{deutsche Textvariante}}{\meta{englische Textvariante}}Dieses Macro gibt \emph{nur} die
Variante in der Erstsprache (bzw. der einzigen gewählten Sprache) aus.
\medskip

\DescribeMacro\cdeen{\meta{deutsche Textvariante}}{\meta{englische Textvariante}}Dieses Macros funktioniert wie
\Macro{ldeen}, richtet seine Ausgabe aber nicht nach der Erstsprache des Dokuments, sondern nach der derzeit umgebenden
Sprache. Es kann also \emph{innerhalb} anderer Sprachmacros verwendet werden und gibt stets die Sprache seines Kontextes
aus.
\medskip

\DescribeMacro\xdeen{\meta{deutsche Textvariante}}{\meta{englische Textvariante}} %{\meta{1. Ersetzungstext}}{\meta{2. Ersetzungstext}}
  Das Macro \Macro{xdeen} funktioniert ähnlich wie \Macro{deen}, kann aber noch eine variable Anzahl weiterer Argumente
  haben. In den jeweiligen Sprachvarianten wird nach dem Vorkommen von ``\verb!@!'' gefolgt von einer Zahl \meta{i}
  gesucht, und jeweils durch das $i$. Argument ersetzt. Das Macro dient dazu, Aufgabenparameter in den verschiedenen
  Sprachvarianten einheitlich zu halten.
  Der Quellcode...
  \begin{lstlisting}
    \documentclass[lang={en,de}]{osgexam}
    % ....
    \xdeen{Gegeben seien die zwei Koordinaten (@1) und
     (@2).}{Given two coordinates (@1) and (@2).}{1,0}{0,3}
  \end{lstlisting}
\noindent  ergibt beispielsweise

\noindent\fbox{
\begin{minipage}{\linewidth}
  \noindent Given two coordinates (1,0) and (0,3).\\\textcolor{blue}{Gegeben seien die zwei Koordinaten (1,0) und (0,3).}
\end{minipage}
}
\medskip

Das Ergebnis der Verwendung von Parameternnummern, die nicht-existierenden Argumenten entsprechen, ist nicht definiert.
\subsection{Titelseite}
\label{sec:macro:title}
Die \thepkg-Klasse legt automatisch eine Titelseite an, so dass sich ein Klausurautor nicht mehr um deren Layout kümmern
braucht.
Die Titelseite erlaubt es, die Klausur in einen A4-Umschlag mit Adressfenster zu stecken, so dass die Personalisierung im
Adressfenster zu sehen ist.

Die Titelseite besteht aus folgenden Elementen:
\begin{itemize}[nosep]
  \item Punkte-/Bewertungstabelle, in die die möglichen Punkte für jede Aufgabe angegeben sind und in die die erreichten Punkte bei
    der Bewertung handschriftlich eingetragen werden können.
  \item Die Personalisierung mit Studenten-ID, Name, Semester und ggf. Sitzplatz
  \item Das Fach, in dem die Klausur geschrieben wird
  \item Eine Liste mit Regeln, die bei der Prüfung zu beachten sind
\end{itemize}
Durch die \thepkg-Klasse werden einige Macros zur Verfügung gestellt, um den Inhalt der Titelseite zu modifizieren.
Sie können nur in der Preamble genutzt werden.

\DescribeMacro\course[\meta{Abkürzung}]{\meta{Kursname}} Der Name der Lehrveranstaltung und ggf.\ eine Abkürzung wird
gesetzt. Falls nötigt, kann der dieser Name mit \Macro{insertCourseName} und die Abkürzung mit \Macro{insertCourseAcronym} in
der Klausur genutzt werden.

\DescribeMacro\term[\meta{Semester-Kurzform}]{\meta{Semester}} Dies definiert das Semester, in dem die Klausur geschrieben wird,
sowie ggf.\ eine Kurzform. Falls benötigt, kann auf das hier definierte Semester in der Klausur mit
\Macro{insertCourseSemester} und auf seine Kurzform mit \Macro{insertCourseSemesterShort} zurückgegriffen werden.
\medskip

\DescribeMacro\solutionTime{\meta{Zeit}} Dies setzt die Lösungszeit, die jedoch auch über die Option \opt{solutiontime}
eingestellt werden kann.

\DescribeMacro{addextrapoints}[\meta{Bezeichnung}]{\meta{Punkte}} Nimmt in den Punktesspiegel Zusatzpunkte auf, die
keiner einzelnen Aufgabe der Klausur zugeordnet werden können, beispielsweise Punkte einer Vorleistung oder Extrapunkte
für Form etc.

\DescribeMacro{addFrontPageRule}{\meta{Regeltext}}
Dieses Macro fügt eine Prüfungsregel auf der Titelseite hinzu. Durch die \thepkg-Klasse sind bereits Standardregeln
vorgegeben, die aber modifiziert werden können. Dazu sind die Regeln intern durchnummeriert, beginnend mit $1$. Mit
Hilfe von \Macro{replaceFrontPageRule}{\meta{Num"-mer der Regel}}{\meta{Regeltext}} kann eine Regel modifiziert werden,
und mit Hilfe von \Macro{removeFrontPageRule}{\meta{Nummer der Regel}} ganz gelöscht werden. Gelöschte Regeln werden
nicht mehr angezeigt, behalten aber ihre Nummer, so dass sich weitere Modifikationen stets auf die ursprüngliche
Nummerierung beziehen.

\subsection{Examsgruppen}
\label{sec:groups}
Die \thepkg-Klasse bietet die Möglichkeit, gezielt bestimmte Bereiche der
Klausur auszuschließen und auf diese Weise verschiedene Gruppen zu ermöglichen.
Die aktuelle Gruppe der Klausur kann im Seriendruck aus der Excel-Datei gelesen
werden, durch das Macro \Macro{setgroup} gesetzt, oder im Singelmodus eingegeben
werden.
Die Abfrage im Singlemodus erfolgt jedoch nur, wenn die Klassenoption
\opt{group} angegeben wurde.

\DescribeEnv{onlygroup}{\meta{Liste von Gruppen}} Verarbeitet den Inhalt der
Umgebung nur, wenn die aktuell gewählte Gruppe in der kommaseparierten \meta{Liste von
  Gruppen} enthalten ist.
\DescribeMacro\setgroup{\meta{Gruppe}} Setzt die aktuelle Gruppe. Dies sollte
nur genutzt werden, wenn kein Seriendruck gewünscht ist.

\section{Einschränkungen, Probleme und Bugs}
\subsection{Einschränkungen und Änderungen gegenüber der \pkg{exam}-Klasse}
Während die Klasse \pkg{exam} die eierlegende Wollmilichsau ist und vieles zulässt, ist \thepkg\ auf einen bestimmten
Einsatzfall festgelegt. Entsprechend können nicht alle Möglichkeiten von \pkg{exam} genutzt werden.
Dieser Abschnitt zählt die wichtigsten Einschränkungen auf:
\begin{itemize}
   \item Das Layout der Titelseite einschließlich der verwendeten \mbox{Punkte-/}""Bewertungstabelle ist vorgegeben und kann (ohne größeren
    Aufwand) nur im Rahmen der Klassenoptionen und der im Abschnitt~\ref{sec:macro:title} beschriebenen Macros geändert werden.
 
    
  \item Das Verhalten von \thepkg\ entspricht dem von \pkg{exam} bei gesetzter Option \opt{addpoints}. Für Fragen, die keine
    Punkte enthalten~-- also in dem weder \Macro{question} noch \env{part}, \env{subpart} oder \env{subsubpart}
    mit einem optionalen Argument genutzt werden, werden stets 0 Punkte angezeigt.
    
  \item \thepkg\ gibt das Layout für Fragen vor, hat einen eigenen Punktemechanismus und ein angepasstes Format für
    die Punktedarstellung. Entsprechend können Punkte nur Zahlen sein. D.\;h. z.\;B., dass eine Prozentangabe für eine
    Aufgabe (wie  auf  Seite 37 der \pkg{exam}-Dokumentation dargestellt) nicht möglich ist. Auch unterstützt \thepkg\
    keine halben Punkte.\footnote{Die Notwendigkeit, halbe Punkte zu vergeben, deutet ohnehin in der Regel auf ein
      schlechtes Klausurdesign hin.}
\end{itemize}

\subsection{Probleme}
Seit Mik\TeX/Mac\TeX\ 2020 funktioniert \pkg{kvoption-patch} nicht mehr, das dazu diente, auch komplexere Klassenoptionen
mit \pkg{keyval}/\pkg{kvoptions} zu parsen. Daher wurde \thepkg\ auf
\pkg{pgfkeys}/\pkg{pgfopts} umgestellt. Die früheren Versionen von \thepkg\
(<0.2) laufen damit nicht auf neueren \TeX-Installationen.

Die \thepkg-Klasse ruft andere Programme auf, namentlich den Python-Interpreter und \file{latexmk}.
Daher muss die Übersetzung stets mit der Aktivierung der Shell-Escape-Funktion vorgenommen werden, also z.B.:\medskip

\noindent\verb!>lualatex -shell-escape myexam.tex!

\medskip
Alternativ kann \texttt{latexmk} mit der im Paket zur Verfügung gestellten \texttt{latexmkrc}-Datei genutzt werden:
\medskip

\noindent\verb!>latexmkrc myexam.tex!

\medskip
\LongWarning{Werden andere \LaTeX-Übersetzungen in dem Verzeichnis durchgeführt, in dem sich die zur Verfügung gestellte
  \file{latexmkrc}-Datei befindet, wird dafür ebenfalls die Shell-Escape-Funktion aktiviert. \textbf{Dies kann ein
    Sicherheitsrisiko darstellen.}}


\subsection{Bugs}
Ich bin sicher, dass die Klasse eine ganz Reihe von Fehlern hat. Zum Zeitpunkt der Erstellung dieser Dokumentation ist
mir jedoch keiner bekannt. Nutzer der Klasse werden gebeten für gefundene Bugs ein Issue auf GitHub anzulegen:
\url{https://github.com/tuc-osg/osg-exam/issues} 

\end{document}
